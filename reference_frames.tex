\documentclass{standalone}
\usepackage{tikz}   %TikZ is required for this to work.  Make sure this exists before the next line

\usepackage{tikz-3dplot}

\usepackage[active,tightpage]{preview}  %generates a tightly fitting border around the work
\PreviewEnvironment{tikzpicture}
\setlength\PreviewBorder{2mm}

\begin{document}

%Angle Definitions
%-----------------

%set the plot display orientation
%synatax: \tdplotsetdisplay{\theta_d}{\phi_d}
\tdplotsetmaincoords{60}{110}

%start tikz picture, and use the tdplot_main_coords style to implement the display 
%coordinate transformation provided by 3dplot
\begin{tikzpicture}[scale=5,tdplot_main_coords]
%define polar coordinates for some vector
%TODO: look into using 3d spherical coordinate system
\pgfmathsetmacro{\rvec}{.8}
\pgfmathsetmacro{\thetavec}{30}
\pgfmathsetmacro{\phivec}{60}
\pgfmathsetmacro{\Px}{0.2}
\pgfmathsetmacro{\Py}{0.346}
\pgfmathsetmacro{\Pz}{0.693}
\pgfmathsetmacro{\b}{0.5}
\pgfmathsetmacro{\d}{0.3}

%set up some coordinates 
%-----------------------
\coordinate (O) at (0,0,0);
\coordinate (Ptikz) at (\Px, \Py, \Pz);
%determine a coordinate (P) using (r,\theta,\phi) coordinates.  This command
%also determines (Pxy), (Pxz), and (Pyz): the xy-, xz-, and yz-projections
%of the point (P).
%syntax: \tdplotsetcoord{Coordinate name without parentheses}{r}{\theta}{\phi}
% \tdplotsetcoord{P}{\rvec}{\thetavec}{\phivec}
%draw figure contents
%--------------------

%draw the main coordinate system axes
\draw[thick,->] (0,0,0) -- (1,0,0) node[anchor=north east]{$e_1$};
\draw[thick,->] (0,0,0) -- (0,1,0) node[anchor=north west]{$e_2$};
\draw[thick,->] (0,0,0) -- (0,0,1) node[anchor=south]{$e_3 = f_3$};

%draw a vector from origin to point (P) 
\draw[-stealth,color=red] (O) -- node [below right] {$ r $} (\Px, \Py, \Pz);

%draw the angle \phi, and label it
%syntax: \tdplotdrawarc[coordinate frame, draw options]{center point}{r}{angle}{label options}{label}
\tdplotdrawarc{(O)}{0.2}{0}{\phivec}{anchor=north}{$\Omega_A t$}


%set the rotated coordinate system so the x'-y' plane lies within the
%"theta plane" of the main coordinate system
%syntax: \tdplotsetthetaplanecoords{\phi}
\tdplotsetthetaplanecoords{\phivec}

% draw the asteroid frame
% \draw[thick,->,tdplot_rotated_coords] (0,0,0) -- (1,0,0) node[anchor=north east]{$f_3$};
\draw[thick,->,tdplot_rotated_coords] (0,0,0) -- (0,1,0) node[anchor=north west]{$f_1$};
% \draw[thick,->,tdplot_rotated_coords] (0,0,0) -- (0,0,1) node[anchor=north west]{$f_2$};

%draw some dashed arcs, demonstrating direct arc drawing
\draw[dashed,tdplot_rotated_coords] (\rvec,0,0) arc (0:90:\rvec);
\draw[dashed] (\rvec,0,0) arc (0:90:\rvec);

%set the rotated coordinate definition within display using a translation
%coordinate and Euler angles in the "z(\alpha)y(\beta)z(\gamma)" euler rotation convention
%syntax: \tdplotsetrotatedcoords{\alpha}{\beta}{\gamma}
\tdplotsetrotatedcoords{\phivec}{\thetavec}{0}

%translate the rotated coordinate system
%syntax: \tdplotsetrotatedcoordsorigin{point}
\tdplotsetrotatedcoordsorigin{(Ptikz)}

%use the tdplot_rotated_coords style to work in the rotated, translated coordinate frame
\draw[thick,tdplot_rotated_coords,->] (0,0,0) -- (\b,0,0) node[anchor=north west]{$b_1$};
\draw[thick,tdplot_rotated_coords,->] (0,0,0) -- (0,\b,0) node[anchor=west]{$b_2$};
\draw[thick,tdplot_rotated_coords,->] (0,0,0) -- (0,0,\b) node[anchor=south]{$b_3$};

% transform the first dumbbell
% rot to main
\tdplottransformrotmain{\d}{0}{0};
\tdplottransformmainscreen{\tdplotresx}{\tdplotresy}{\tdplotresz};
\pgfmathsetmacro{\DMonescreenx}{\tdplotresx};
\pgfmathsetmacro{\DMonescreeny}{\tdplotresy};

\tdplottransformrotmain{-\d}{0}{0};
\tdplottransformmainscreen{\tdplotresx}{\tdplotresy}{\tdplotresz};
\pgfmathsetmacro{\DMtwoscreenx}{\tdplotresx};
\pgfmathsetmacro{\DMtwoscreeny}{\tdplotresy};

% transform P to screen coordinates and then add
\tdplottransformmainscreen{\Px}{\Py}{\Pz};
\pgfmathsetmacro{\Pscreenx}{\tdplotresx};
\pgfmathsetmacro{\Pscreeny}{\tdplotresy};

\draw[ultra thick, color=blue, tdplot_rotated_coords] (-\d, 0, 0) -- (\d, 0, 0);

\shade[ball color=red, tdplot_screen_coords] (\Pscreenx, \Pscreeny) circle (0.03);
\shade[ball color=blue,tdplot_screen_coords] ($(\Pscreenx, \Pscreeny) + (\DMonescreenx, \DMonescreeny) + (0, 0.01)$) circle (0.03);
\shade[ball color=blue,tdplot_screen_coords] ($(\Pscreenx, \Pscreeny) + (\DMtwoscreenx, \DMtwoscreeny) + (0, -0.01)$ ) circle (0.03);

%WARNING:  coordinates defined by the \coordinate command (eg. (O), (P), etc.)
%cannot be used in rotated coordinate frames.  Use only literal coordinates.  

%draw some vector, and its projection, in the rotated coordinate frame
% \draw[-stealth,color=blue,tdplot_rotated_coords] (0,0,0) -- (.2,.2,.2);
% \draw[dashed,color=blue,tdplot_rotated_coords] (0,0,0) -- (.2,.2,0);
% \draw[dashed,color=blue,tdplot_rotated_coords] (.2,.2,0) -- (.2,.2,.2);

%show its phi arc and label
% \tdplotdrawarc[tdplot_rotated_coords,color=blue]{(0,0,0)}{0.2}{0}{45}{anchor=north west,color=black}{$\phi'$}

%change the rotated coordinate frame so that it lies in its theta plane.
%Note that this overwrites the original rotated coordinate frame
%syntax: \tdplotsetrotatedthetaplanecoords{\phi'}
% \tdplotsetrotatedthetaplanecoords{45}

%draw theta arc and label
% \tdplotdrawarc[tdplot_rotated_coords,color=blue]{(0,0,0)}{0.2}{0}{55}{anchor=south west,color=black}{$\theta'$}

\end{tikzpicture}

\end{document}
